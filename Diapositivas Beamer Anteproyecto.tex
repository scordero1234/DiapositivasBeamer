\documentclass{beamer}

% Tema de la presentación
\usetheme{Madrid}
\usepackage{ragged2e} 
\documentclass{article}
\usepackage{pgfgantt}
\usepackage{geometry}
\usepackage{pgfgantt}


\begin{document}

\title{Análisis Comparativo de Selenium y Appium para Pruebas de Regresión en la Aplicación Móvil  de la Cooperativa de Ahorro y Crédito Jardín Azuayo  ”JAMOVIL”: Calidad ,Eficiencia, Cobertura y Experiencia del Usuario}
\author{Cristian Ivan Idrovo Tapia, Santiago David Cordero Crespo} 
\date{\today}

\begin{frame}
  \titlepage
\end{frame}



\begin{frame}{Tabla de Contenido}
  \tableofcontents
\end{frame}

\section{Introducción}


\begin{frame}{Introducción}
\justify
En la actualidad el desarrollo y mantenimiento de aplicaciones móviles se ha convertido en un desafío constante para los desarrolladores, por otra parte se ha potenciado el uso de herramientas tales como Appium y Selenium que permiten automatizar las pruebas de regresión garantizando que las nuevas implementaciones no afecten las características previamente establecidas.
\newline
\newline
La aplicación móvil financiera de la cooperativa de ahorro y crédito Jardin Azuayo ”JAMOVIL” es una solución tecnológica que brinda una experiencia eficiente y accesible para sus usuarios, facilitando operaciones financieras y transacciones. Aplicación que se encuentra en constante evolución para mantener una ventaja competitiva en el mercado financiero.

\end{frame}


\section{Justificación}

\begin{frame}{Justificación}
\justify
El análisis comparativo entre Selenium y Appium, permitirá identificar cual de ellas es la adecuada para garantizar la calidad, eficiencia y experiencia del usuario realizar pruebas de regresión en aplicaciones móviles.\newline
\newline
El uso de estas herramientas ayudará a los desarrolladores a acelerar la liberación de nuevas versiones, lo que mejora la eficiencia del equipo de desarrollo optimizando los tiempos de pruebas de calidad de software y experiencia de usuario
\end{frame}

\section{Objetivos}
\subsection{General}
\begin{frame}{Objetivo General }
  \begin{columns}
    \begin{column}{0.5\textwidth}
     Comparar y Evaluar la calidad , eficiencia, cobertura y experiencia del usuario al realizar pruebas de regresión en la aplicación móvil financiera "JAMOVIL", utilizando las herramientas de automatización Selenium y Appium, para obtimizar el proceso de pruebas mejorando la calidad del software
   
    \end{column}
    \begin{column}{0.5\textwidth}
      \begin{itemize}
          \item Que voy hacer ?
          \begin{itemize}
              \item Comparar y Evaluar la calidad , eficiencia, cobertura y experiencia del usuario al realizar pruebas de regresión en la aplicación móvil financiera "JAMOVIL"
          \end{itemize}
          \item Como lo voy hacer ?
          \begin{itemize}
              \item Utilizando las herramientas de automatización Selenium y Appium
          \end{itemize}
          \item ¿Para qué lo voy a hacer?
           \begin{itemize}
              \item Para obtimizar el proceso de pruebas mejorando la calidad del software.
          \end{itemize}         
          \end{itemize}
    \end{column}
  \end{columns}
\end{frame}

\subsection{Objetivo Específico 1 }
\begin{frame}{Objetivo Específico 1 }
  \begin{columns}
    \begin{column}{0.5\textwidth}
    \justify
      Realizar una revisión exhaustiva de la literatura científica y técnica sobre pruebas de regresión en aplicaciones móviles financieras, enfocándose en estudios, investigaciones determinar cuál de las dos herramientas puede brindar resultados de pruebas más rápidos y exhaustivos para mejorar la calidad de Software.
    \end{column}
    \begin{column}{0.55\textwidth}
      \begin{itemize}
          \item Que voy hacer ?
          \begin{itemize}
              \item Realizar una revisión exhaustiva de la literatura científica y técnica sobre pruebas de regresión en aplicaciones móviles.
          \end{itemize}
          \item Como lo voy hacer ?
          \begin{itemize}
              \item enfocándose en estudios, investigaciones 
          \end{itemize}
          \item ¿Para qué lo voy a hacer?
          \begin{itemize}
              \item determinar cuál de las dos herramientas puede brindar resultados de pruebas más rápidos y exhaustivos para mejorar la calidad de Software.
          \end{itemize}
      \end{itemize}
    \end{column}
  \end{columns}
\end{frame}
\subsection{Objetivo Específico 2 }
\begin{frame}{Objetivo Específico 2 }
  \begin{columns}
    \begin{column}{0.5\textwidth}
    \justify
     Definir el conjunto de requerimientos y criterios de evaluación utilizando herramientas de automatización Selenium y Appium en las pruebas de regresión de la aplicación móvil financiera "JAMOVIL" para evaluar y comparar la eficiencia, calidad, cobertura de casos de prueba y experiencia del usuario 
    \end{column}
    \begin{column}{0.5\textwidth}
      \begin{itemize}
          \item Que voy hacer ?
          \begin{itemize}
              \item Definir el conjunto de requerimientos y criterios de evaluación
          \end{itemize}
          \item Como lo voy hacer ?
          \begin{itemize}
              \item utilizando herramientas de automatización Selenium y Appium en las pruebas de regresión de la aplicación móvil financiera "JAMOVIL"
          \end{itemize}
          \item ¿Para qué lo voy a hacer?
          \begin{itemize}
                  \item Para evaluar
y comparar la eficiencia, calidad,
cobertura de casos de prueba y
experiencia del usuario
          \end{itemize}
      \end{itemize}
    \end{column}
  \end{columns}
\end{frame}
\subsection{Objetivo Específico 3}
\begin{frame}{Objetivo Específico 3 }
  \begin{columns}
    \begin{column}{0.5\textwidth}
    \justify
      Implementar la automatización de pruebas con Selenium y Appium, asegurando la configuración adecuada en el ambiente de pruebas y dispositivos móviles, para recolectar información que será utilizada en el proceso de análisis y resultados. 
    \end{column}
    \begin{column}{0.5\textwidth}
      \begin{itemize}
          \item Que voy hacer ?
          \begin{itemize}
              \item  Implementar la automatización de pruebas con Selenium y Appium.
          \end{itemize}
          \item Como lo voy hacer ?
          \begin{itemize}
              \item Asegurando la configuración adecuada en el ambiente de pruebas y dispositivos móviles.
          \end{itemize}
         \item ¿Para qué lo voy a hacer?
         \begin{itemize}
              \item Para recolectar información que será utilizada en el proceso de análisis y resultados.
          \end{itemize}
      \end{itemize}
    \end{column}
  \end{columns}
\end{frame}

\subsection{Objetivo Específico 4}
\begin{frame}{Objetivo Específico 4 }
  \begin{columns}
    \begin{column}{0.5\textwidth}
    \justify
      Medir y comparar las pruebas de regresión obtenida con Selenium y Appium, considerando el número de casos de prueba ejecutados y la profundidad de la cobertura en diferentes escenarios de prueba, para determinar cual de las dos herramientas ofrece una mayor cobertura de pruebas. 
    \end{column}
    \begin{column}{0.5\textwidth}
      \begin{itemize}
          \item Que voy hacer ?
          \begin{itemize}
              \item  Medir y comparar las pruebas de regresión obtenida con Selenium y Appium
          \end{itemize}
          \item Como lo voy hacer ?
          \begin{itemize}
              \item considerando el número de casos de prueba ejecutados y la profundidad de la cobertura en diferentes escenarios de prueba
          \end{itemize}
         \item ¿Para qué lo voy a hacer?
         \begin{itemize}
              \item Para determinar cual de las dos herramientas ofrece una mayor cobertura de pruebas
          \end{itemize}
      \end{itemize}
    \end{column}
  \end{columns}
\end{frame}
\subsection{Objetivo Específico 5}
\begin{frame}{Objetivo Específico 5 }
  \begin{columns}
    \begin{column}{0.5\textwidth}
    \justify
      Identificar fortalezas y debilidades de cada herramienta de automatización Selenium y Appium, considerando factores como la complejidad de la implementación, la facilidad de mantenimiento y la escalabilidad, para tomar las decision sobre cual herramienta es la mas adecuada para realizar el proceso de pruebas en la JAMOVIL 
 
    \end{column}
    \begin{column}{0.5\textwidth}
    
      \begin{itemize}
      
          \item Que voy hacer ?
          \begin{itemize}
              \item Identificar fortalezas y debilidades de cada herramienta de automatización Selenium y Appium
          \end{itemize}
          \item Como lo voy hacer ?
          \begin{itemize}
              \item considerando factores como la complejidad de la implementación, la facilidad de mantenimiento y la escalabilidad
          \end{itemize}
         \item ¿Para qué lo voy a hacer?
         \begin{itemize}
              \item para tomar las decision sobre cual herramienta es la mas adecuada para realizar el proceso de pruebas en la JAMOVIL
          \end{itemize}
      \end{itemize}
    \end{column}
  \end{columns}
\end{frame}



\section{Metodología}

\begin{frame}{Metodología}
\justify
Se ejecutará una revisión de la literatura científica y técnica sobre pruebas de regresión en la aplicaciones móvil “JAMOVIL”, enfocándose en estudios relevantes. Posterior a ello, se analizarán y evaluarán las herramientas de automatización Selenium y Appium, identificando sus características, ventajas y desventajas en el contexto específico de pruebas de regresión.
\newline
\newline
Se definirán requerimientos y criterios de evaluación para las pruebas de regresión de la aplicación "JAMOVIL", considerando aspectos como eficiencia, calidad, cobertura de casos de prueba y experiencia del usuario durante el proceso de pruebas.
La implementación de la automatización de pruebas involucra configurar el ambiente de pruebas con dispositivos móviles, emuladores o simuladores,  para recolectar datos relevantes para el análisis.
\end{frame}


\section{Cronograma}

\begin{frame}{Cronograma}
\begin{figure}[h]
\centering
\includegraphics[width=1\textwidth]{cronograma.png}
\caption{Cronograma de actividades}
\label{FIG1:1}
\end{figure}
\end{frame}

\section{Presupuesto}
\begin{frame}{Presupuesto}
    
\end{frame}

\end{document}

